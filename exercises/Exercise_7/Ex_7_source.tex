\begin{aufgabe}[Newton Loop]

To show how important the choice of a good starting value $x_0$ is for the Newton-Raphson method, we consider the following function $f(x)=x^3 - 2x + 2$ for which we want to calculate $x$ so that $f(x)=0$. 

\begin{teilaufgabe}

\item Calculate the derivative of f(x).
\item Write down the Newton-Raphson iteration formula for the function $f(x)$, substituting $f(x)$ and $f'(x)$ accordingly.
\item Using the Newton-Raphson iteration formula, calculate the first five iteration results $x_1, x_2, x_3, x_4, x_5$ if the value 0 is used for $x_0$.
\item Using the Newton-Raphson iteration formula, calculate the first five iteration results $x_1, x_2, x_3, x_4, x_5$ if the value -1 is used for $x_0$.
\item Using the Newton-Raphson iteration formula, calculate the first five iteration results $x_1, x_2, x_3, x_4, x_5$ if the value -2 is used for $x_0$.
\end{teilaufgabe}

\end{aufgabe}




\begin{aufgabe}[Inverse Calculation via Brute Force Newton]

In the lecture, we learned that the inverse of $b$ can be calculated using the following iteration formula:~$x_{n+1} = x_n (2 - x_n b)$. In this task, we implement the brute force approach, which calculates the initial guess $x_0$ for the inverse determination using the Newton-Raphson approach.
To do this, we first define the following nomenclatures and assumptions:
(1) we denote the value whose inverse is to be determined as $b$, 
(2) we assume that $b \in [u,o]$, 
(3) we denote the number of calculated iterations of the Newton-Raphson method as $iN$, 
(4) we calculate $x_0$ using the linear function $x_0=m*t+y$, and (5) we denote the amount of sampling points as $s$.

Note: The tasks a - e must not be implemented homomorphically.

\begin{teilaufgabe}

\item Implement the auxiliary function $h(t,m,y)$, which returns $x_0$
\item Implement the function \textbf{Newton($t,x_0,iN$)}, which returns $x_{iN}$
\item Implement the function \textbf{Error($t,x_0,iN$)}, which returns the difference of $t^{-1}$ and \textbf{Newton$(t,x_0,iN)$}   
\item Implement the function \textbf{ErrorTotal($m,y,o,u,s,iN$)}, which returns $\sum_{i=0}^{s}$~\textbf{Error$(t_i,h(t_i,m,y),iN)^2$} where $t_i= u + \frac{o-u}{s} * i$
\item Use the above implementations to find the good values for $m \in \{-1,-0.9,...,0.9,1\}$ and $y \in \{-1,-0.9,...,0.9,1\}$ for the calculation of the division function on the interval $[10, 20]$ with $s = 20, iN = 3$. 
\item Use the found values for $m$ and $y$ to homomorphically compute the inverse of $15$.
\end{teilaufgabe}

\end{aufgabe}




\begin{aufgabe}[Square Root Calculation via Brute Force Newton]
In the lecture, we learned that the square root of $a$ can be calculated by first using the Newton-Raphson approach to approximate $\frac{1}{\sqrt{a}}$ and then multiply this result with $a$. Implement the calculation of the square root analogous to Task 2.
\end{aufgabe}
