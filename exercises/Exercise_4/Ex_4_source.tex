\begin{aufgabe}[More examples for encoding and decoding]
Use the algorithms from the lecture and apply CKKS encoding and decoding in the following two subtasks.
Write down your intermediate results for $z, p, m, h$ and the final result.

\begin{teilaufgabe}
\item Input: $v= \left(\begin{array}{c} 23 \\ 15  \end{array}\right)$
\end{teilaufgabe}

\begin{teilaufgabe}
\item Input: $v= \left(\begin{array}{c} 4 \\ -3  \end{array}\right)$
\end{teilaufgabe}

\begin{teilaufgabe}
\item What do you observe when looking at the result from the decoding algorithm? How does this observation correspond to the original input for the encoding?
\end{teilaufgabe}

\begin{teilaufgabe}
\item Bonus: Implement the encoding and decoding algorithms in a programming language of your choice.
\end{teilaufgabe}

\end{aufgabe}



\begin{aufgabe}[Polynomial interpolation using the Vandermonde matrix]
As already mentioned in the lecture, the CKKS encoding algorithm is basically a polynomial interpolation operation.
To do this in the general case you can use the Vandermonde matrix and write the problem like this:


\[
V(\vec{x}) * \vec{z} \stackrel{!}{=} \vec{y}
\]

$\vec{x}$ are the $x$ coordinates for which we evaluate the resulting polynomial and $\vec{y}$ are the $y$ coordinates we want to receive ($\vec{x},\vec{z},\vec{y}$ are $n$-dimensional vectors).
We now need to solve for $\vec{z}$ to get the coefficients of the polynomial that interpolates the points.

Find a polynomial that interpolates the points $(4,-6), (-3,2), (5,1)$ using the above equation.


\end{aufgabe}
