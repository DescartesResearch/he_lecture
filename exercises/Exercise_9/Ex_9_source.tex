\begin{aufgabe}[Homomorphic Taylor Series]

Another approach to calculating homomorphic functions is polynomial approximations such as the Taylor series. The Taylor series approximates a function $f$ around a development point $x_0$ by determining the corresponding polynomial $TS_f(x) = \sum_{n = 0}^m \frac{f^{(n)}(x_0)}{n!}(x-x_0)^n$. Where $f^{(n)}$ stands for the n-th derivative of the function $f$.
In the following, we will compare how the Taylor series behaves in comparison to the Newton-Raphson method for calculating the root and inverse function.

Note: For the comparison you don't need to implement the approaches homomorphically.

\begin{teilaufgabe}
    \item To approximate the root function $sqr$ using the Taylor series, we must determine the corresponding polynomial $TS_{sqr}$ around the development point $x_{sqr}$.
    \begin{teilaufgabe}
        \item[i)] Calculate the polynomial $TS_{sqr}$ in general if the highest allowed degree is  $x^3$.
        \item [ii)] Determine the concrete polynomial $TS_{sqr}$ for the development point $x_0 = 10$ if the highest allowed degree is $x^3$.
        \item[iii)] Using the previously determined polynomial, calculate the root function for the values $\{ 5.0, 5.1, ...14.9, 15.0 \}$. Next, calculate the root function for the same values using the Newton-Raphson method (you can assume the following interval for the Newton-Raphson method: $ [ 5; 15 ] $). Finally, plot the results of the two methods for calculating the root for the values $\{ 5.0, 5.1, ... 14.9, 15.0 \}$ in the same diagram.
    \end{teilaufgabe}
    \item Repeat task a) and increase the permitted degree in steps of one up to 8.
    \item To approximate the root function $inv$ using the Taylor series, we must determine the corresponding polynomial $TS_{inv}$ around the development point $x_{inv}$. 
    \begin{teilaufgabe}
        \item[i)] Calculate the polynomial $TS_{inv}$ in general if the highest allowed degree is  $x^3$.
        \item [ii)] Determine the concrete polynomial $TS_{inv}$ for the development point $x_0 = 10$ if the highest allowed degree is $x^3$.
        \item[iii)] Using the previously determined polynomial, calculate the inverse function for the values $\{ 5.0, 5.1, ...14.9, 15.0 \}$. Next, calculate the root function for the same values using the Newton-Raphson method (you can assume the following interval for the Newton-Raphson method: $ [ 5; 15 ] $). Finally, plot the results of the two methods for calculating the inverse for the values $\{ 5.0, 5.1, ... 14.9, 15.0 \}$ in the same diagram.
    \end{teilaufgabe}
    \item Repeat task c) and increase the permitted degree in steps of one up to 8.
\end{teilaufgabe}


\end{aufgabe}
