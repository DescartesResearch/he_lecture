\begin{aufgabe}[Modular arithmetic]
The lecture introduced the concept of congruence.
For the following exercises you may need to research the basic properties of the congruence relationship.

\begin{teilaufgabe}
\item Given that $4x \equiv 1 \mod{9}$, find $x$.
    
\end{teilaufgabe}

\begin{teilaufgabe}
\item Given that $12x \equiv 8 \mod{20}$, find one solution for $x$.
    
\end{teilaufgabe}

\begin{teilaufgabe}
\item Consider the same equation as in b) \\
Are there more solutions for $x$? If yes: Find all possible solutions.
    
\end{teilaufgabe}

\begin{teilaufgabe}
\item Find the last digit of $7^{100}$.

\end{teilaufgabe}

\begin{teilaufgabe}
\item In year $N$, the 300th day of the year is a Tuesday. In year $N+1$, the 200th day is also a Tuesday. On what day of the week did the 100th day of the year $N-1$ occur?
  
\end{teilaufgabe}

\begin{teilaufgabe}
\item If $n!$ denotes the product of integers $1$ through $n$, what is the remainder when $(1! + 2! + 3! + 4! + 5! + 6! + ...)$ is divided by 9?
    
\end{teilaufgabe}

\end{aufgabe}
