\begin{aufgabe}[Rings]
Show that the following structures satisfy the ring axioms and are therefore rings.

\begin{teilaufgabe}
\item The definition of polynomial rings from the lecture.
    
\end{teilaufgabe}

\begin{teilaufgabe}
\item The set of 2-by-2 square matrices ($M_2(R)$) with entries in a ring $R$ together with matrix multiplication and matrix addition.

\[
M_2(R) = \left\{ \begin{pmatrix}a & b \\ c & d\end{pmatrix} \Big| a,b,c,d \in R \right\}
\]
\end{teilaufgabe}
\end{aufgabe}



\begin{aufgabe}[Reducibility]
In the lecture we defined the term reducibility in the context of polynomials.
For this task we give a more concrete definition:

\emph{Let $R$ be an integral domain (nonzero commutative ring) and let $p$ be a non-zero non-unit in $R[X]$, we say that $p$ is reducible over $R$ if we can factor $p = gh$ where both $g$ and $h$ are non-units.
Otherwise we say that $p$ is irreducible over $R$.}

The definition of a unit is:

\emph{A unit of a ring is an invertible element for the multiplication of the ring. That is, an element $u$ of a ring $R$ is a unit if there exists $v$ in $R$ such that 
$v u = u v = 1$ 
where $1$ is the multiplicative identity.}

Now decide for the following polynomials if they are reducible over the corresponding domains


\begin{teilaufgabe}
\item $2x+2$ over $\mathbb{Z}$
    
\end{teilaufgabe}

\begin{teilaufgabe}
\item $2x+2$ over $\mathbb{R}$
    
\end{teilaufgabe}

\begin{teilaufgabe}
\item  $x^2-5$ over $\mathbb{R}$
    
\end{teilaufgabe}

\begin{teilaufgabe}
\item  $x^2-5$ over $\mathbb{Q}$
    
\end{teilaufgabe}



\begin{aufgabe}[Polynomial division]
Give a short description on how to calculate the remainder when given any polynomial $p$ and a modulus $m$.
You can describe the process by writing down the steps (pseudo code like).
\end{aufgabe}



\begin{aufgabe}[Root of unity]
Proof the following theorems:

\begin{teilaufgabe}
\item If $x$ is a $n$-th root of unity, then so is $x^k$, where $k \in \mathbb{Z}$.
\end{teilaufgabe}

\begin{teilaufgabe}
\item If $z$ is a $n$-th root of unity and $a \equiv b \pmod{n}$ then $z^a = z^b$.
\end{teilaufgabe}

\begin{teilaufgabe}
\item If $z$ is a $n$-th root of unity and $z^a = z^b$, then $a \equiv b \pmod{n}$ may be false.
\end{teilaufgabe}

\end{aufgabe}
